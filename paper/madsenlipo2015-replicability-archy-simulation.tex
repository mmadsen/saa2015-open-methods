
\documentclass[preprint,times,authoryear,10pt]{elsarticle}

%% The amssymb package provides various useful mathematical symbols
\usepackage{amssymb,amsmath}


%%%%%%%%%% Remove the following before submission %%%%%%%%%%%%%%%%%%

\usepackage{mathspec,xltxtra,xunicode}
%\usepackage{unicode-math}
%\defaultfontfeatures{Scale=MatchLowercase}
\setmainfont[Mapping=tex-text,Numbers=OldStyle]{Times New Roman}
%\setmainfont[Ligatures=TeX,Numbers=OldStyle]{Minion Pro}
\setsansfont[Mapping=tex-text]{ITC Legacy Sans Std Medium}
\setmonofont{Bitstream Vera Sans Mono}
%\setmathfont(Digits,Latin,Greek)[Script=Math,Uppercase=Italic,Lowercase=Italic]{Minion Math Semibold}
%\setmathfont[range={\mathbfup->\mathup}]{MinionMath-Bold.otf}
%\setmathfont[range={\mathbfit->\mathit}]{MinionMath-Bold.otf}
%\setmathfont[range={\mathit->\mathit}]{MinionMath-Bold.otf}

%%%%%%%%%% Remove the above before submission %%%%%%%%%%%%%%%%%%

%% The amsthm package provides extended theorem environments
%% \usepackage{amsthm}

%% The lineno packages adds line numbers. Start line numbering with
%% \begin{linenumbers}, end it with \end{linenumbers}. Or switch it on
%% for the whole article with \linenumbers after \end{frontmatter}.
\usepackage{lineno}
\usepackage{graphicx}
\usepackage{xspace}
\usepackage{bm}
\usepackage{longtable}
\usepackage{hyphenat}
\usepackage{lipsum}
\usepackage{url}
\usepackage{outlines}
\usepackage{diss-macros}
\usepackage[section,ruled]{algorithm}
\usepackage{algorithmic}
\usepackage{boxedminipage}
\usepackage[xetex,bookmarks=true,linkcolor=blue,hyperfootnotes=false,breaklinks=true,citecolor=blue,colorlinks=true]{hyperref}
\usepackage{sistyle}
\SIthousandsep{,}

\journal{Electronic Symposium on Open Methods in Archaeology}

% Pandoc toggle for numbering sections (defaults to be off)

% Pandoc header


\begin{document}

\begin{frontmatter}


\title{Tools for Transparency and Replicability of Simulation in Archaeology}

\author{Mark E. Madsen}
\address{Department of Anthropology, Box 353100, University of Washington, Seattle WA, 98195 USA}
\ead{mark@madsenlab.org}
\ead[url]{http://madsenlab.org}

\author{Carl P. Lipo}
\address{Department of Anthropology and IIRMES, 1250 Bellflower Blvd, California State University at Long Beach, Long Beach CA, 90840 USA}
\ead{Carl.Lipo@csulb.edu}
\ead[url]{http://lipolab.org}


\begin{abstract}
Simulation is an increasingly central tool across many theoretical frameworks but especially in evolutionary archaeology. Simulation and numerical analysis is routinely employed in hypothesis tests and model development. Simulations, however, have a well-deserved reputation as difficult to replicate and test, and it is rare that researchers beyond the authors can build upon a previously published simulation study. To improve replicability, and to make our work accessible, we employ standard tools and scripted analyses, and engage a standard software development toolchain. We describe our workflow as a contribution to best practices for simulation in archaeology.

\end{abstract}

\begin{keyword}
simulation \sep replicability \sep open methods \sep archaeology
\end{keyword}


\end{frontmatter}

\section{Abstract (for ref while
writing)}\label{abstract-for-ref-while-writing}

Simulation is an increasingly central tool across many theoretical
frameworks but especially in evolutionary archaeology. Simulation and
numerical analysis is routinely employed in hypothesis tests and model
development. Simulations, however, have a well-deserved reputation as
difficult to replicate and test, and it is rare that researchers beyond
the authors can build upon a previously published simulation study. To
improve replicability, and to make our work accessible, we employ
standard tools and scripted analyses, and engage a standard software
development toolchain. We describe our workflow as a contribution to
best practices for simulation in archaeology.

\section{Notes}\label{notes}

\begin{outline}[enumerate]
    \1 Simulations are almost never reused and built upon, and simulated data are not reused for multiple studies
        \2 Should we expect to reuse code?  Yes.  
        \2 Should we expect to reuse simulated data?  Yes, under suitable circumstances.
        \2 Example of cultural transmission models - we reimplement the same models over and over.  Why?   
    \1 Difficulties in sharing, reusing, and replicating simulations
        \2 Baz
\end{outline}

Reasons to reuse:

\begin{enumerate}
\def\labelenumi{\arabic{enumi}.}
\itemsep1pt\parskip0pt\parsep0pt
\item
  Iterating on model development -- writing code isn't the point,
  building models is. At least in the exploratory stages, to the extent
  that we can use existing (tested) code, we can focus on the model,
  rather than its implementation.\\
\item
  Understanding other people's work -- this doesn't have to be strict
  replication of results, because almost none of us spend time doing
  that. But I often see a model in a published paper, and want to
  explore its limits, or what would happen if the authors had used
  different combinations of parameters. In essence, I want to do model
  development, but starting from someone else's work.\\
\item
  Pedagogy -- having students who aren't programmers reimplement most
  models is not productive. Except in a context where the point is to
  train students to write their own complex models, having students
  build upon existing and well documented models is probably a better
  way to teach both the models, and the use of simulation in
  contemporary science.
\end{enumerate}

Reuse for any of these reasons requires that we develop and use
simulation models using open methods and open tools, with an explicit
eye to their reusability. Replicability is a hot issue in computational
science, psychology and other social sciences (but not so much anthro),
and medical research, but replicability (in the broad sense) as a goal
doesn't solve everything. Transparency of operation and transparency in
how the resulting data are generated by the model are crucial.

\subsection{Elements of the current
approach}\label{elements-of-the-current-approach}

\begin{enumerate}
\def\labelenumi{\arabic{enumi}.}
\itemsep1pt\parskip0pt\parsep0pt
\item
  Everything is version controlled, except \textbf{output data} which
  are too large to fit in a version control system.
\item
  Each simulation run is assigned a globally unique identifier (UUID)
\item
  All simulations are run in batch mode, with parameters clearly
  specified on each command line.
\item
  Each simulation run is assigned a randomly chosen seed for the random
  number generator, and this seed is stored WITH all output data
  records.
\item
  All output data records have key parameters stored in the record, and
  the simulation command used to invoke it.\\
\item
  Keep a ``data dictionary'' up to date with each release of the
  simulation software, describing the output data and metadata it
  creates.
\item
  Ensure that output data used in analysis always comes from a named,
  tagged release of the software, not an arbitrary checkin to source
  code control, so that the exact software situation can be duplicated.
\end{enumerate}

\subsection{Globally Unique Identifiers for Simulation
Runs}\label{globally-unique-identifiers-for-simulation-runs}

Use true UUIDs (universally unique identifiers) that can be created in a
decentralized fashion (i.e., without contacting a central authority).
RFC1422 \url{http://tools.ietf.org/html/rfc4122.html} provides a
specification for creating several types of UUIDs, combining accurate
datetime and random elements, often with something like the hardware
network address (MAC) of the machine generating the UUID. Simple
libraries exist in many programming languages to generate such values.
In the SeriationCT project (and others), we use the standard Python UUID
library, and employ UUID-1 as the identifier variant, since it
incorporates the hardware MAC address of the node executing the
simulation.

The use of UUIDs as simulation identifiers virtually guarantees that a
specific set of output can be tied back to the code and parameters used
to create it, so long as the simulation metadata are available.

\subsection{Treating Simulation Sets as
Experiments}\label{treating-simulation-sets-as-experiments}

Discuss the named experiment directories, and layout of directories for
simulation and analysis, storage of raw data and what happens when you
mess up and need to start over.

\section{Discussion}\label{discussion}

Lorem ipsum dolor sit amet, consectetur adipiscing elit. Vestibulum
viverra est est. Proin eget tellus metus. Aenean ac tortor pharetra
libero ultricies sagittis. Nulla facilisi. Cras tincidunt interdum
tellus, quis consectetur nunc facilisis nec. Sed fermentum erat a ligula
posuere quis semper risus ullamcorper. Morbi vel tincidunt augue. Nam
dolor ipsum, sagittis quis dignissim eu, pulvinar sed magna. In interdum
magna eu orci facilisis congue. Cras a tellus et lorem sagittis viverra.
Donec risus lectus, mollis at dignissim viverra, dapibus a nulla.
Vivamus porttitor scelerisque turpis, eget lobortis orci auctor eget.
Donec ultricies enim ac augue porttitor convallis. Pellentesque nisl
lorem, consequat a facilisis in, ornare sed lorem. In luctus, elit ac
mattis dapibus, lacus elit varius tortor, vel sollicitudin massa nisl id
massa. Ut sit amet nibh a sem egestas sollicitudin. Vestibulum
scelerisque, dui at tincidunt accumsan, ipsum enim feugiat neque, vel
interdum turpis lectus sed nisi. Nullam ultrices sodales sem, et
placerat nunc euismod eu. Duis leo lacus, semper quis eleifend vitae,
viverra ut nisl. Vestibulum ante ipsum primis in faucibus orci luctus et
ultrices posuere cubilia Curae; Proin rutrum eleifend est, id tempor
velit viverra sed. Test citations:
\citep{BR1985, CF1981, national2012Assessing, Ewens2004, xie2013dynamic}

\section{Acknowledgements}\label{acknowledgements}

Lorem Ipsum


%% References with bibTeX database:

\bibliographystyle{elsarticle-harv}
\bibliography{madsenlipo2015-replicability-archy-simulation}









\end{document}

%%
%% End of file `elsarticle-template-2-harv.tex'.
