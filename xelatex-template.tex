
\documentclass[preprint,times,authoryear,10pt]{elsarticle}

%% The amssymb package provides various useful mathematical symbols
\usepackage{amssymb,amsmath}


%%%%%%%%%% Remove the following before submission %%%%%%%%%%%%%%%%%%

\usepackage{mathspec,xltxtra,xunicode}
%\usepackage{unicode-math}
%\defaultfontfeatures{Scale=MatchLowercase}
\setmainfont[Mapping=tex-text,Numbers=OldStyle]{Times New Roman}
%\setmainfont[Ligatures=TeX,Numbers=OldStyle]{Minion Pro}
\setsansfont[Mapping=tex-text]{ITC Legacy Sans Std Medium}
\setmonofont{Bitstream Vera Sans Mono}
%\setmathfont(Digits,Latin,Greek)[Script=Math,Uppercase=Italic,Lowercase=Italic]{Minion Math Semibold}
%\setmathfont[range={\mathbfup->\mathup}]{MinionMath-Bold.otf}
%\setmathfont[range={\mathbfit->\mathit}]{MinionMath-Bold.otf}
%\setmathfont[range={\mathit->\mathit}]{MinionMath-Bold.otf}

%%%%%%%%%% Remove the above before submission %%%%%%%%%%%%%%%%%%

%% The amsthm package provides extended theorem environments
%% \usepackage{amsthm}

%% The lineno packages adds line numbers. Start line numbering with
%% \begin{linenumbers}, end it with \end{linenumbers}. Or switch it on
%% for the whole article with \linenumbers after \end{frontmatter}.
\usepackage{lineno}
\usepackage{graphicx}
\usepackage{xspace}
\usepackage{bm}
\usepackage{longtable}
\usepackage{hyphenat}
\usepackage{lipsum}
\usepackage{url}
\usepackage{outlines}
\usepackage{diss-macros}
\usepackage[section,ruled]{algorithm}
\usepackage{algorithmic}
\usepackage{boxedminipage}
\usepackage[xetex,bookmarks=true,linkcolor=blue,hyperfootnotes=false,breaklinks=true,citecolor=blue,colorlinks=true]{hyperref}
\usepackage{sistyle}
\SIthousandsep{,}

\journal{Electronic Symposium on Open Methods in Archaeology}

% Pandoc toggle for numbering sections (defaults to be off)
$if(numbersections)$
$else$
\setcounter{secnumdepth}{0}
$endif$

% Pandoc header
$for(header-includes)$
$header-includes$
$endfor$


\begin{document}

\begin{frontmatter}


\title{Tools for Transparency and Replicability of Simulation in Archaeology}

\author{Mark E. Madsen}
\address{Department of Anthropology, Box 353100, University of Washington, Seattle WA, 98195 USA}
\ead{mark@madsenlab.org}
\ead[url]{http://madsenlab.org}

\author{Carl P. Lipo}
\address{Department of Anthropology and IIRMES, 1250 Bellflower Blvd, California State University at Long Beach, Long Beach CA, 90840 USA}
\ead{Carl.Lipo@csulb.edu}
\ead[url]{http://lipolab.org}


\begin{abstract}
Simulation is an increasingly central tool across many theoretical frameworks but especially in evolutionary archaeology. Simulation and numerical analysis is routinely employed in hypothesis tests and model development. Simulations, however, have a well-deserved reputation as difficult to replicate and test, and it is rare that researchers beyond the authors can build upon a previously published simulation study. To improve replicability, and to make our work accessible, we employ standard tools and scripted analyses, and engage a standard software development toolchain. We describe our workflow as a contribution to best practices for simulation in archaeology.

\end{abstract}

\begin{keyword}
simulation \sep replicability \sep open methods \sep archaeology
\end{keyword}


\end{frontmatter}

$body$


%% References with bibTeX database:

\bibliographystyle{elsarticle-harv}
\bibliography{$biblio-files$}









\end{document}

%%
%% End of file `elsarticle-template-2-harv.tex'.
